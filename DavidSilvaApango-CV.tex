\documentclass[twoside,letter,openright,10pt]{report}

\usepackage[utf8]{inputenc}
\usepackage[spanish]{babel}
\usepackage{colortbl}
\usepackage{enumitem}
\usepackage{fontawesome}
\usepackage{makecell}
\usepackage{multirow}

\setlength{\textwidth}{190mm}
\setlength{\textheight}{270mm}
\setlength{\oddsidemargin}{-15mm}
\setlength{\evensidemargin}{-15mm}
\setlength{\topmargin}{-30mm}

\pagestyle{empty}


\begin{document}
\begin{Huge}
\centering
Maestro en Física / Científico de Datos\\
\end{Huge}
\begin{large}
\centering
Maestro en Ciencias David Silva Apango\\
\end{large}
\begin{table}[hbt!]
\begin{tabular}{p{40mm}p{140mm}}

%multi-column{number cols}{align}{text}
\multicolumn{2}{l}{\faEnvelope\ \texttt{ddsilvaa06@gmail.com}}
\\
\multicolumn{2}{l}{\faMobile\ +52 2224-736702}
\\
\multicolumn{2}{l}{\faGlobe\ https://davidsa06.github.io/}
\\
\multicolumn{2}{l}{\faGithub\ DavidSA06}
\\
\multicolumn{2}{l}{\faLinkedinSquare\ David Silva Apango}
\\
\multicolumn{2}{l}{8 A Sur 6513 Colonia Loma linda, Puebla, Pue. C.P. 72477}
\\
%%%%%%%%%%%%%%%%%%%%%
%Objetivo Profesional
%%%%%%%%%%%%%%%%%%%%%
\multicolumn{2}{c}{\cellcolor{black} \textcolor{white}{Objetivo Profesional}}
\\
\\
& He estudiado el Doctorado en Física Aplicada en la Facultad de Ciencias Físico-Matemáticas (FCFM) de la Universidad Autónoma de Puebla (BUAP). Busco colaborar en una posición como Data Scientist por mi formación en matemáticas y mi experiencia en divulgación científica.
\\
\\
%%%%%%%%%%%%%%%%%%%%%%
%Información Académica
%%%%%%%%%%%%%%%%%%%%%%
\multicolumn{2}{c}{\cellcolor{black} \textcolor{white}{Información Académica}}
\\
\\
\textbf{2017/2021} & Doctorado en Ciencias \textbf{(Física Aplicada)}.
\\
& Benemérita Universidad Autónoma de Puebla.
\\
\textbf{2015/2017} & Maestría en Ciencias \textbf{(Física Aplicada)}.
\\
& Benemérita Universidad Autónoma de Puebla.
\\
\textbf{2009/2015} & Licenciatura en \textbf{Física Aplicada}.
\\
& Benemérita Universidad Autónoma de Puebla.
\\
%%%%%%%%%%%%%%%%%%%%%%%%
%Experiencia Profesional
%%%%%%%%%%%%%%%%%%%%%%%%
\multicolumn{2}{c}{\cellcolor{black} \textcolor{white}{Experiencia Laboral}}
\\
\\
\textbf{2019/2020} & \textbf{Coasesor de tesis} de Licenciatura en Física Aplicada
\\
\textbf{2019/2021} & \textbf{Consejero Universitario} del sector alumno de la Facultad de Ciencias Físico-Matemáticas
\\
\textbf{2014/2021} & Miembro del \textbf{Capítulo Estudiantil de la Sociedad de Óptica de América (OSA)}, FCFM-BUAP. Con los cargos desempeñados :\\
& \vspace{-2mm} \begin{itemize}[noitemsep,nolistsep]
\item Presidente.
\item Vicepresidente.
\item Tesorero.
\vspace{-4mm}
\end{itemize}
\\
%%%%%%%%%%%%%%%%%%%%%%%%%%
%Experiencia Universitaria
%%%%%%%%%%%%%%%%%%%%%%%%%%
\multicolumn{2}{c}{\cellcolor{black} \textcolor{white}{Experiencia Universitaria}}
\\
\\
\textbf{6/11-10-2019} &LXXII Congreso Nacional de Física. Villahermosa, Tabasco. Participante.
\\
\textbf{13/17-05-2019} &Segundo Congreso Internacional Luz Ciencia y Arte (IICILCA). Puebla, Puebla. \textbf{Organizador}.
\\
\textbf{23-07-2014/22-08-2014} & Verano de la Investigación Científica de la Academia Mexicana de Ciencias en el Centro de Investigaciones en Óptica. Participante.
\\
\\
\multicolumn{2}{c}{\cellcolor{black} \textcolor{white}{Información Adicional}}
\\
\\
\textbf{Personal} &  Fecha de nacimiento: 6-9-1990
\\
& Disponibilidad para viajar, \textbf{visa} de Estados Unidos.
\\
& Software: Python, Assembly(PICs), MATLAB, LabView, Minitab, Latex, C.
\\
& Hobbies: Modelismo estático, divulgación científica y correr.
\\
\textbf{Técnica}
& Idiomas:
\begin{itemize}[noitemsep,nolistsep]
\item Español (Nativo)
\item Inglés (Intermedio)
\item Japonés (N5)
\vspace{-4mm}
\end{itemize}
\\
& Cursos:
\begin{itemize}[noitemsep,nolistsep]
\item Profesional de Git y GitHub (Platzi)
\item Python Intermedio (Platzi)
\item Funciones en C (Platzi)
\item Python (FCFM)
\vspace{-4mm}
\end{itemize}
\end{tabular}
\end{table}
\end{document}
