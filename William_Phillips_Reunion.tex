\documentclass[10pt,letterpaper]{report}
\usepackage[utf8]{inputenc}
\usepackage[spanish]{babel}
\usepackage{amsmath}
\usepackage{amsfonts}
\usepackage{amssymb}
\begin{document}
Resumen de la reunión con el Dr. William D. Phillips y el Dr. Luis A. Orozco.
\\
La reunión empezó con una breve presentación tanto del Dr. Phillips como del Dr. Orozco. Después el Dr. Phillips hizo recomendaciones sobre a los congresos, siendo una de ellas al respecto de los gafetes al decir que prescindiéramos de los cordones y usáramos una pinza que se coloque en medio del gafete haciendo un agujero con un cuchillo ya que al ser físicos deberíamos tener uno, esto se hace con el propósito de que el gafete sea visible para las demás personas. También realizó un comentario sobre las mochilas que se reparten en los congresos ya que no deberíamos usar la mochila del congreso actual porque no se destaca entre todos los asistentes, lo mejor sería guardar la mochila y usar una del congreso al que asistimos anteriormente.
\\
El Dr. Phillips nos motivó a hacerle preguntas por medio de premios que repartiría a los que participaran: Dos tarjetas que contenían las constantes fundamentales de la física, la primera contenía los valores recomendados actualmente mientras que la segunda abarcaba los valores que se usarían en la próxima convención.
\\
Las preguntas realizadas fueron:
\\
¿Qué oportunidades tienen los estudiantes de posgrado de Estados Unidos después de egresar en cuanto a oportunidades de trabajo?
Ambos Doctores concuerdan que algo común entre el personal académico de las universidades es que no atraen a nuevos estudiantes contándoles que los lugares disponibles de trabajo como investigador son reducidos en comparación con la cantidad de estudiantes que egresan pero esto no debe ser causa de preocupación porque la industria actualmente está solicitando personal con conocimientos en física, matemáticas, ingeniería, especialmente en el área de la óptica cuántica. Además el Dr. Orozco mencionó que es posible trabajar en la venta de equipo y material de laboratorios porque las personas encargadas de los laboratorios prefieren a vendedores que conozcan su lenguaje y entiendan cómo funcionan los aparatos que venden.
¿Cómo prepara sus presentaciones?
El Dr. Phillips considera que necesita mucho tiempo para elaborar sus presentaciones, en caso de dar una presentación sobre un tema que no había expuesto antes tarda aproximadamente diez horas para preparar una hora de exposición y si la presentación es sobre un tema que ya expuso entonces el tiempo de preparación disminuye a dos o tres horas. El Dr Phillips reconoce que esa labor no es su fuerte y que si alguien no tiene la suficiente dedicación para impartir clases debería considerar trabajar en la industria.
Cuando eres estudiante de licenciatura o incluso de maestría, te acercas a un profesor para pedirle que te dé un tema de tesis en el que puedas trabajar. ¿Cómo es la transición de ser el estudiante que pide un tema a ser el profesor que propone el tema y las ideas?
¿Puedes decirme cómo es que un niño aprende a leer? No. En general, las personas aprendemos a hacer algo porque vemos que otra persona lo hace. Si tú le sonríes a alguien, aprenderá a sonreír, si le hablas a alguien, aprenderá a hablar, si le lees a alguien aprenderá a leer, si amas a alguien, aprenderá a amar. Lo mismo sucede con proponer ideas para una investigación. Si al estudiante le propones ideas y las discutes con él, eventualmente aprenderá a hacer esto. Esta madurez del estudiante, casi siempre se da cuando se encuentra al final del doctorado o en el post-doctorado. La clave es que cada vez dejes que el estudiante tome más iniciativa en la proposición de ideas.
¿Cuando era niño pensó que ganaría el premio Nobel?
El Dr. Phillips nos narró que cuando era niño él no pensaba en ganar el premio Nobel pero que se planteó alcanzar ese objetivo a los 10 años y conforme fue avanzando en sus estudios poco a poco se dio cuenta de que esa era meta muy lejana y poco realista. También nos contó que cuando era un niño le regalaron un microscopio en el que observaba varios objetos además de crear su propio juego de química combinando varios artículos de limpieza y que está feliz de no haber creado una reacción química que resultara peligrosa.
¿Cómo puede balancear su vida científica, familiar y social?
No todo en la vida es física, las relaciones con otras personas son muy importantes, lleva 47 años de casado, piensa que no vale la pena comprometer a la familia por el trabajo científico, es muy triste ver a compañeros que son unos completos extraños en su casa. Preferiría regresar a su casa a cenar con su familia y contarles un cuento a sus hijos antes de dormir que ganar el premio Nobel.
¿Por qué decidió ser físico?
Porque al tener mala memoria decidió estudiar algo dentro del área de las ciencias, el problema era que la química es olorosa, la biología asquerosa pero la física era más limpia, no se sentía muy listo para estudiar matemáticas pero la física al tener matemáticas se sintió complacido.
¿Cómo podemos llamar la atención de los niños para que se interesen más en la ciencia?
Se puede atraer a los niños con colores, debe dejarse que los niños toquen los experimentos porque para ellos es importante percibir con sus propias manos. No debe intentarse cambiar la mente de los niños porque con su propia curiosidad ellos se dirigen solos hacia la ciencia. Debe evitarse dar explicaciones aburridas. Es más importante que las personas vean que los experimentos son divertidos a que entiendan bien los conceptos.
¿Hay alguna relación entre la ciencia y la iglesia?
No la hay, ambas son muy distintas y no están relacionadas, ya que en la ciencia puedes comprobar hipótesis o tienes evidencias para demostrar la existencia y en la iglesia no sabes quién es dios o si en verdad existe o no, sin embargo es importante creer en algo siempre y cuando sepas diferenciar entre el contenido que ofrece la biblia y la ciencia.
¿Cuál fue su peor error (en el laboratorio)?
Seguidamente de pedirnos que no nos riéramos decidió contarnos una anécdota en la que pensó que trabajar en trampas magneto-ópticas podría ser interesante. Después de diseñar un aparato muy grande y costoso mandó a maquinarlo, este proceso tardó tanto tiempo que en cuanto llegó el aparato al laboratorio él y sus compañeros ya habían pensado en varios temas interesantes en los cuales mantenerse ocupados por lo que este se quedó guardado mientras todos se dedicaban a otras ocupaciones, más tarde decidió entregarlo a otras personas que podrían darle un uso. Al Dr. Phillips no le satisfizo contar esta historia debido a que no aprendió mucho de este error por lo que también nos contó sobre un experimento con molasas ópticas que son un arreglo en el que tres pares de haces láser que se contra propagan perpendicularmente entre sí. Gracias a este arreglo podían enfriar un grupo de átomos que al enfriarse oscilarían poco pero que al apagar los láseres pensaron que los átomos se despegarían violentamente entre sí como si se tratase de una explosión por lo que una buena forma de estudiar este comportamiento sería colocar un detector arriba de la molasa óptica. Al apagar los láseres el sensor no detectaba nada a pesar de las repeticiones del experimento. Por lo que decidieron dejar el sensor abajo. En esa ocasión el sensor detectó una enorme cantidad de átomos, resultó que el grupo de átomos se enfrió más de lo previsto y en vez de estallar calló directo al sensor. Además del resultado del experimento el Dr. Phillips aprendió a no centrarse en la misma manera de pensar demasiado.
¿Cómo ganar el premio Nobel?
Las personas que se concentran en ganar un premio nobel no lo ganarán, se trata de seguir estudiando y aprendiendo, eso en sí es un premio con la dedicación podría venir un nobel.
¿Qué es lo más importante que debe tomarse en cuenta en una carrera en ciencia?
Lo más importante es nunca perder el interés y gusto por lo que haces, podrías tener un trabajo en el que tienes talento pero del cual no te sientas satisfecho, por eso es importante amar el trabajo que realizas. 
¿Cómo divulgar ciencia a niños?
Lo primero que hay que hacer al exponer a cualquier otra persona es disminuir las expectativas ya mostrar que la ciencia es divertida. Bajo la consideración del Dr. Phillips es mejor que las personas se diviertan en una presentación a que entiendan completamente bien los conceptos. Puso como ejemplo una exposición que suele hacer ya que trabaja con temperaturas bajas necesita el nitrógeno líquido, recomendó que obtuviésemos una licencia para manejarlo para evitar lastimarnos. Después de verter nitrógeno líquido en un recipiente mete uno tras otro varios globos inflados. Por el enfriamiento del nitrógeno líquido los globos quedan planos al sacarlos del recipiente y es en ese momento en el que los arroja al público como si fueran frisbees.
¿Cuáles son sus pasatiempos?
La fotografía y cantar góspel en el coro de la Iglesia
¿Cuál es la parte más difícil de ser físico y creyente al mismo tiempo?
No hay dificultad porque la ciencia te explica como funciona el mundo y te abre la mente mientras la biblia es un libro que te enseña cómo convivir con los demás, es la palabra de Dios pero no es un libro de ciencia. Aunque haya personas que no sean respetuosas con las creencias religiosas los creyentes debían respetar a los demás. Al final concluyó diciendo que cuando uno va a la iglesia debe abrir el corazón, abrir los brazos y abrir la mente.
Después de ganar el premio Nobel ¿Cuáles son sus metas en la ciencia?
Seguir impartiendo conferencias y realizar simulaciones relacionadas a la física cuántica.

\end{document}