\documentclass[twoside,letter,openright,10pt]{report}

\usepackage{CJKutf8}
\usepackage[utf8]{inputenc}

\usepackage{colortbl}
\usepackage{enumitem}
\usepackage{fontawesome}
\usepackage{makecell}
\usepackage{multirow}

\setlength{\textwidth}{190mm}
\setlength{\textheight}{270mm}
\setlength{\oddsidemargin}{-15mm}
\setlength{\evensidemargin}{-15mm}
\setlength{\topmargin}{-30mm}

\pagestyle{empty}


\begin{document}
\begin{CJK}{UTF8}{goth}
\begin{huge}
\centering
科学修士号 (応用物理学)\\\\
\end{huge}
\begin{Large}
\centering
デビッド・シルバ修士課程\\
\end{Large}
\begin{table}[hbt!]
\begin{tabular}{p{40mm}p{140mm}}

%multi-column{number cols}{align}{text}
\multicolumn{2}{l}{\faEnvelope\ \texttt{ddsilvaa06@gmail.com}}
\\
\multicolumn{2}{l}{\faMobile\ +52 2224-736702}
\\
\multicolumn{2}{l}{\faGlobe\ https://davidsa06.github.io/}
\\
\multicolumn{2}{l}{\faGithub\ DavidSA06}
\\
\multicolumn{2}{l}{\faLinkedinSquare\ David Silva Apango}
\\
\multicolumn{2}{l}{ぷエブラ市、プエブラ州、墨西哥}
\\
%%%%%%%%%%%%%%%%%%%%%
%専門的な目標
%%%%%%%%%%%%%%%%%%%%%
\multicolumn{2}{c}{\cellcolor{black} \textcolor{white}{専門的な目標}}
\\
& 私はBUAPの物理数理科学部 (FCFM) で応用物理学の博士号で勉強しました。 私は数学、プログラミング、センサーデータ処理を学びしたため、アナリストのポジションに応募しました。 応用物理学の学士論文をアドバイスをしたり、通俗科学の章を主導したりして、プレゼンテーションのスキルを身につけました。
\\
%%%%%%%%%%%%%%%%%%%%%%
%学術情報
%%%%%%%%%%%%%%%%%%%%%%
\multicolumn{2}{c}{\cellcolor{black} \textcolor{white}{学術情報}}
\\
\textbf{2017/2021} & 科学博士号 \textbf{(応用物理学)}.
\\
& Benemérita Universidad Autónoma de Puebla(BUAP)。 光電子研究所。
\\
\textbf{2015/2017} &  \textbf{(応用物理学)}。
\\
& Benemérita Universidad Autónoma de Puebla (BUAP)。 光電子研究所。
\\
\textbf{2009/2015} & \textbf{応用物理学}の学士。
\\
& Benemérita Universidad Autónoma de Puebla (BUAP)。 光電子研究所。
\\
%%%%%%%%%%%%%%%%%%%%%%%%
%専門的の経験
%%%%%%%%%%%%%%%%%%%%%%%%
\multicolumn{2}{c}{\cellcolor{black} \textcolor{white}{専門的の経験}}
\\
\textbf{2019/2020} & 応用物理学の\textbf{卒業論文アドバイザー}。
\\
\textbf{2022/-} & Ktdra 出版社の\textbf{コンテンツクリエータ}。
\\
%%%%%%%%%%%%%%%%%%%%%%%%%%
%大学時代での経験
%%%%%%%%%%%%%%%%%%%%%%%%%%
\multicolumn{2}{c}{\cellcolor{black} \textcolor{white}{大学時代での経験}}
\\
\textbf{2014/2022} & OPTICA (旧 OSA)学生支部のメンバー、FCFM-BUAP。 肩書き:\\

& \vspace{-2mm}
\begin{itemize}[noitemsep,nolistsep]
\item 支部長。
\item 副支部長。
\item 会計係。
\vspace{-4mm}
\end{itemize}
\\
\textbf{2019/2021} & \textbf{大学カウンセラー}、 FCFM学生部門の名誉大学評議。
\\
\textbf{6/11-10-2019} &第62回全国物理学会議。ビヤエルモサ市, タバスコ州。 参加者、 \textbf{旅行主催者}。
\\
\textbf{13/17-05-2019} &第2回国際光、科学、芸術会議 (IICILCA)。 ぷエブラ市、プエブラ州。 \textbf{主催者}。
\\
\textbf{23-07-2014/22-08-2014} & 墨西哥科学学院の科学研究の夏、光学研究センター。レオンし、グアナフアト州 。 参加者。
\\
%%%%%%%%%%%%%%%%%%%%%%%%%%
%追加情報
%%%%%%%%%%%%%%%%%%%%%%%%%%
\multicolumn{2}{c}{\cellcolor{black} \textcolor{white}{追加情報}}
\\
\textbf{個人的} &  生年月日: 千九百九十年九月六日
\\
& 社会技能: 指導、チームワークとプレゼンテーション能力.
\\
& 旅行の可能性、 \textbf{米国ビザ}.
\\
\textbf{基礎学力}
& ソフトウェア: Microsoft Office, Python, SQL, Assembly(PICs), MATLAB, LabView, Minitab, \LaTeX{}.
\\
& 言語:
\begin{itemize}[noitemsep,nolistsep]
\item スペイン語 (母語話者).
\item 英語 (中上級).
\item 日本語 (N5).
\vspace{-4mm}
\end{itemize}
\\
& コース:
\begin{itemize}[noitemsep,nolistsep]
\item Git and GitHub Professional Course (Platzi).
\item Professional Python Course (Platzi).
\item Database Fundamentals, SQL (Platzi).
\item Tableau (Platzi).
\item Python Course (FCFM).
\vspace{-4mm}
\end{itemize}
\end{tabular}
\end{table}
\end{CJK}
\end{document}
